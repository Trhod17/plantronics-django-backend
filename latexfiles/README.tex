\documentclass{article}

\usepackage[dvipsnames]{xcolor}
\usepackage{etoolbox}
\usepackage{hyperref}
\usepackage{minted}
\usemintedstyle{monokai}
\usepackage[explicit]{titlesec}
\usepackage{setspace}
\usepackage[english]{babel}
\usepackage[a4paper, total={6in, 8in}]{geometry}

\hypersetup{colorlinks,
	linkcolor=green,
	linktoc=all}

\makeatletter % change only the display of \thepage, but not \thepage itself:
\patchcmd{\ps@plain}{\thepage}{\textcolor{white}{\thepage}}{}{}
\makeatother

{\large }
\titleformat{\section}
{\normalfont\normalsize\bfseries\centering}{}{0em}{\MakeUppercase{#1}}

\definecolor{bg}{HTML}{282828}

\colorlet{LightRubineRed}{RubineRed!70}
\colorlet{Mycolor1}{green!10!orange}
\definecolor{Mycolor2}{HTML}{00F9DE}

\setcounter{tocdepth}{3}



\title{Instructions to setup and configure `Plantronixs' web service }
\author{Thomas Rhodes}



\begin{document}
	\pagestyle{plain}
	
	\pagecolor{darkgray}
	\color{white}%
	\maketitle
	
	\tableofcontents
	
		
	\begin{spacing}{2.5}
	\end{spacing}
	\part{For Linux Users}

	\section{Python setup}
		\subsection{Check Python}
	Check to see if you have a compatible version of python installed, either by running
	\mintinline[bgcolor=bg]{console}{$ python --version} or 
	\mintinline[bgcolor=bg]{console}{$ python3 --version} 
	if successful it should look something similar to the following.
	\begin{minted}[bgcolor=bg]{bash}
		[trhod17@k9 ~] $ python --version
		Python 3.10.2  
	\end{minted}
	If if worked move to step 2, if not continue to step 1.2.
	
		\subsection{Installing Python}
		
		Install python by running the correct command for your distrobution
		
		\begin{description}
			\item[Debian or Ubuntu based systems:]
				\item\mintinline[bgcolor=bg]{console}|$ sudo apt install python3|
			\item[Arch based systems:]
				\item\mintinline[bgcolor=bg]{console}|$ sudo pacman -S python|
				\item\mintinline[bgcolor=bg]{console}|$ sudo pamac install python|
			\item[RHEL based systems:]
				\item\mintinline[bgcolor=bg]{console}|$ sudo dnf install python3 |
		\end{description}
		
		Once installed retry step 1.1 to make sure its installed properly
	
	\section{Setting Up The Environment}
		\subsection{Getting The Source Code}
			Go to \href{https://github.com/Trhod17/tafe_plant_water_timer_app_backend}{the Github repo (click here)} 
			and Download as a ZIP, then extract to your desired location or clone the repo by using \\
\mintinline[bgcolor=bg]{console}|$ git clone https://github.com/Trhod17/tafe_plant_water_timer_app_backend|
		 
		 \subsection{Setting Up Python Virtual Environment}	
			 Now you can either create a new python virtual environment, \href{https://docs.python.org/3/tutorial/venv.html}{Guide Here}, or use the existing environment saved in the repo by running 
			 \begin{description}
			 	\item[For Bash Consoles]
			 	\item\mintinline[bgcolor=bg]{console}|$ source django_tafe/bin/active |
			 	\item[For Zsh and other consoles]
			 	\item\mintinline[bgcolor=bg]{console}|$ . django_tafe/bin/active |
			 \end{description}
		 	in the folder containing the source code.
		 	
		  	\subsubsection{Created New Python Virtual Environment}
		 		If you have chosen to create new python virtual environment you next step is to 
		 		install the required packages by navigating to the directory containing the source code and running \newline \mintinline[bgcolor=bg]{console}|$ pip install -r requirements.txt|
			
			\subsection{Run Django}
				Once you have finished the above steps run
			\begin{minted}[bgcolor=bg]{console}
(env) [trhod17@k9 django-tafe]$ python manage.py runserver 
Watching for file changes with StatReloader 
Performing system checks... 
System check identified no issues (0 silenced). April 03, 2022 - 08:10:55 
Django version 4.0.3, using settings ‘plantronics.settings’
Starting development server at http://127.0.0.1:8000/ 
Quit the server with CONTROL-C. 
			\end{minted} 
			And the website should now be accessible via http://127.0.0.1:8000
		
		\subsection{Proj 2 Specifications}
			Please refer to the other pdf document located in the same directory as this file called RHODES\_459274825\_Proj2.pdf, on how to test this web service against the specifications for Proj 2, Examples are also given for clarity.
			
		\subsection{Proj 2, 18}
			This step is for when you are up to number 18 in the other document,
			go to roughly line 174, and change the value of `burst' rate to 100/second then save, open a new console in the project root directory and or in your IDE, open the python environment as specified above and run \mintinline[bgcolor=bg]{console}|$ pythom manage.py test|
			and this should run 8 test cases which should each container 3 tests 
	
	
	\begin{spacing}{3}
	\end{spacing}
	\part{For Windows Users}
	
	\begin{framed}
		Disclaimer: \newline While the steps below SHOULD work, I do not own a windows machine, nor do i have a virtual machine setup to test these instructions
		therefore it is of great preference to the author/ developer that this is ran in a Linux environment, as intended. This can be done within windows by installing a Linux distro as an App/program via WSL or by way of a virtual machine 
	\end{framed}
	
	
	\section{Install Python}
	Go to \href{https://www.python.org/downloads/release/python-3104/}{www.python.org/downloads/release/python-3014/} and select the
	appropriate installer for your version of Windows. Download and run the installer, This will install Python on your machine so that we may proceed.
	
	\section{Setup Environment}
	Go to \href{https://github.com/Trhod17/tafe_plant_water_timer_app_backend}{the Github repo (click here)} and download as a ZIP and extract to
	desire location, or use git clone to clone the repo and open in desired ide.
	
	\section{Create And Setup Virtual Environment}
	
	\subsection{Create Virtual Environment}

	Now we must create a virtual environment, so activate a terminal in your IDE and type
	\mintinline[bgcolor=bg]{console}|$ python -m venv env| or \mintinline[bgcolor=bg]{console}|$ python3 -m venv env|, you will see a new folder created called 'env', we need this so don't delete it. 
	\newline
	\newline
	In a terminal within your IDE type the following command:
	\mintinline[bgcolor=bg]{console}|$ env\bin\activate| to activate the Python Virtual Environment needed in order to setup this web service.
	\newline
	\newline
	This Python Virtual Environment holds all the packages and dependencies necessary
	for this web service to run, it works independently of any pre-existing globally
	installed packages, as a self contained file system.
	
	\subsection{Setup Virtual Environment}
	Now we must install all our package dependencies by typing into the terminal
	\newline\mintinline[bgcolor=bg]{console}|$ pip install -r requirements.txt|.
	
	\subsection{Launching Django}
	In the same terminal, with our (env), type \mintinline[bgcolor=bg]{console}|$python manage.py runserver| to start the web service. The web service will be accessible by default via http://127.0.0.1:8000. You should get a similar terminal readout to:

		\begin{minted}[bgcolor=bg]{console}
		(env) $ python manage.py runserver
		Watching for file changes with StatReloader
		Performing system checks...
		
		System check identified no issues (0 silenced).
		April 03, 2022 - 08:10:55
		Django version 4.0.3, using settings 'plantronics.settings'
		Starting development server at http://127.0.0.1:8000/
		Quit the server with CONTROL-C.
	\end{minted}

	\subsection{Proj 2 Specifications}
	Please refer to the other pdf document located in the same directory as this file called RHODES\_459274825\_Proj2.pdf, on how to test this web service against the specifications for Proj 2, Examples are also given for clarity.
	
	\subsection{Proj 2, 18}
	This step is for when you are up to number 18 in the other document,
	go to roughly line 174, and change the value of `burst' rate to 100/second then save, open a new console in the project root directory and or in your IDE, open the python environment as specified above and run \mintinline[bgcolor=bg]{console}|$ pythom manage.py test|
	and this should run 8 test cases which should each container 3 tests  
	\newline
	\newline
	\newline
	\newline

	\begin{framed}
			\centering END OF DOCUMENT.
			\newline
			\newline 
			\centering THANKS FOR READING!
	\end{framed}
	
\end{document}
